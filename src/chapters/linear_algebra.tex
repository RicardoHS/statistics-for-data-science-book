This chapter will focus on the very basis of mathematical knowledge and linear algebra, but builded from the very beginning.

\section{Number Sets}
\begin{itemize}
    \item Natural $\mathbb{N} = \{1, 2, 3, \dots\}$
    \item Integer $\mathbb{Z} = \{\dots, -2, -1, 0, 1, 2, \dots\}$
    \item Rational $\mathbb{Q} = \{r | r= \frac{m}{n},$ where $m,n \in \mathbb{Z}, n \neq 0\}$
    \item Irrational $\mathbb{P} = \mathbb{R} \setminus \mathbb{Q} = \{\pi, \sqrt{2}, e, \dots \}$
    \item Real $\mathbb{R} = \mathbb{P} \cup \mathbb{Q}$
    \item Complex $\mathbb{C} = \{ z = a +bi, $ where $a,b\in \mathbb{R}, i^2=-1\}$
\end{itemize}

\section{Fundamental theorem of Algebra}
Any n$^{th}$ degree polynomial, such as,
\[ r(x) = x^n + \alpha_1 x^{n-1} + \alpha_2 x^{n-2} + \dots + \alpha_{n-1}x + \alpha_n \]
has $n$ roots in $\mathbb{C}$ allowing multiplicity. Where
\begin{itemize}
    \item A root is a value such as $r(x)=0$
    \item Multiplicity means that a root can be repeat. ex: A root repeated two times is called a root with multiplicity of two.
\end{itemize} 

We can use those roots to factorize a polynomial. For example, for a polynomial of grade two we can use the formula
$\frac{-b \pm \sqrt{b^2 - 4ac}}{2a}$
\[ r(x) = ax^2 + bx + c \Leftrightarrow r(x) = a(x-x_+)(x-x_-) \]

There is also a formula for cubic polynomials (grade three)\\

For polynomials of greather grade, it can be used the Ruffini's Rule\\
TODO

\section{Linear Equations}
