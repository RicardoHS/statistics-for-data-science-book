This chapter will focus in the procedures to analyze a phenomena of which some
information is unknown. Until now, we had analyzed problems and experiments in
which the whole structure of the problem is known before. But, what happen when
we couldn't know the probabilities related with all events? There is any
approach to overcome this? Someone could just think, "Well, lets perform the
experiment $n$ times and let's see what happen."\\

This approach is the one we are going to follow in this chapter, and in it, it's
analyzed the problems and benefits of performing such approach. The only prior
we have to take in to account is that \textbf{all random phenomena follows an
underlying probability function}. With this in mind. Let's start defining in a
formal way "perform the experiment $n$ times"

\subsection{Parametric Family of Distributions}
Let $F$ be a p.d.f. with parameter $\theta$\\

Usually, for some random experiments, the trials could give us some information
about the shape of the underlying probability function $F$ of the experiment. In
practice, $F$ is not exactly know, but we know that his parameter $\theta$ model
his shape. So, we can know that $F$ belongs to a set or \textit{parametric
family of distributions} $\{F(\;\cdot\;;\theta)\; |\; \theta \in \Theta\}$.\\

Because our objective is to know more about the real $F$ we need to perform some
procedures in order to \textit{infer} the real value of $\theta$. If we achieve
to infer the real value, or at least, an approximately one, then we could find,
a good approximation of $F$ inside his family of distributions. In order to
extract this information about $F$ we need to perform independent realizations
of the experiment related with $F$. With this extracted knowledge we could
perform inference over $F$ or his parameter $\theta$.\\

The fact that we perform independent repetitions of the experiment means that,
for each repetition of the experiment, we have an associated r.v. with the same
distribution $F$. This is called, \textit{simple random sample}.

\subsection{Types of estimation (Inference)}
\todo[inline]{Review the taxonomy and refactor to be the same in the whole book}
\begin{itemize}
    \item Model Based (exact, approximated)
    \item Sampling Based
    \item Bayesian Based
\end{itemize}

\subsection{Random Sample}
A \textit{simple random sample} of a r.v. $X$ with distribution $F$ is a
collection of r.v. $(X_1, X_2,\dots,X_n)$ that are \textbf{independent and
identically distributed (i.i.d.)} and all of them follows the same distribution
$F$\\

Because the random variables are independent, the c.d.f. of the sample is,
\[ F_X(x_1,x_2,\dots,x_n) = F(X_1)F(x_2)\dots F(x_n) \]

\subsection{Statistic}
A statistic $T$ is any measurable function $T: (\mathbb{R}^n,\mathcal{B}^n)
\rightarrow (\mathbb{R}^k,\mathcal{B}^k)$, where $k$ is the dimension of the
statistic. For example:
\begin{itemize}
    \item $T_1(X_1,X_2,\dots,X_n) = \frac{1}{n}\sum_{i=1}^{n}X_i \triangleq \bar{X}_n$, this is called \textbf{Sample Mean}
    \item $T_1(X_1,X_2,\dots,X_n) = \frac{1}{n-1}\sum_{i=1}^{n} (X_i-\bar{X})^2 \triangleq S'^2_n$, this is called \textbf{Sample Variance}
    \item $T_1(X_1,X_2,\dots,X_n) = \min \{ X_1,\dots,X_n \} \triangleq X_{(1)}$
    \item $T_1(X_1,X_2,\dots,X_n) = \max \{ X_1,\dots,X_n \} \triangleq X_{(n)}$
    \item $T_1(X_1,X_2,\dots,X_n) = \sum_{i=1}^{n} \log{X_i}$
    \item $T_1(X_1,X_2,\dots,X_n) = (X_{(1)}, X_{(n)})$
\end{itemize} 
All these statistics have dimension $k=1$ except the last with $k=2$. The
distribution induced by the statistic $T$ is called the \textit{sampling
distribution} of $T$, since it depends on the distribution of the sample.

\subsection{Estimators or Point Estimators}
An estimator is an statistic that tries to estimate the value of a parameter
$\theta$ that belongs to the distribution of the variable $X$. Thus,\\

Let $X \sim F(\cdot\;;\;\theta)$ where $\theta$ is a parameter vector with
possibles values in the \textit{parameter space} $\Theta$. An estimator
$\hat{\theta}_n$ of $\theta$ is a statistic $\hat{\theta}_n(X_1,\dots,X_n)$ with
values also in $\Theta$ 

To clarify with a example, in the real world many r.v. follows a normal
distribution with mean $\mu \in \mathbb{R}$ and variance $\sigma^2 \in
\mathbb{R}^+$ that are unknown. Because of this, it's usually assumed that the
distribution of a r.v. belong to a normal family of distributions
$\{\mathcal{N}(\mu,\sigma^2):\mu\in\mathbb{R},\sigma^2\in \mathbb{R^+} \}$.
Knowing this, it is said that the \textbf{sample mean} $\bar{X}$ and the
\textbf{sample variance} $S^2$ are good estimators of the distribution mean
$\mu$ and distribution variance $\sigma^2$.

\begin{table}[H]
    \centering
    \begin{tabular}{ll}
    Population Parameter    & Sample Estimator                          \\
    $p = P(X=1)$            & $\hat{p}_n = \frac{1}{n}\sum_{i=1}^n X_i$ \\
    $\mu = \mathbb{E}[X]$   & $\bar{X}_n = \frac{1}{n}\sum_{i=1}^n X_i$ \\
    $\sigma^2 = V[X]$       & $S'^2_n = \frac{1}{n-1}\sum_{i=1}^{n}
    (X_i-\bar{X})^2 $
    \end{tabular}
\end{table}

\section{Exact inference under Normal Distributions}
\textbf{Sampling distributions in Normal Populations}\\
The sample mean $\bar{X}$ and sample variance $S^2$ estimators play an important
role in statistical inference, since both are “good” estimators of $\mu$ and
$\sigma^2$, respectively. As a consequence, it is important to obtain their
sampling distributions in order to know their random behaviors. We will do so
under the assumption of normal populations.

\subsection{Expected value and Variance of the sample mean}
Let $\bar{X}_n = \frac{1}{n}\sum_{i=1}^n X_i$ be the sample mean. Then, it
holds,
\[ \mathbb{E}[\bar{X}_n] = \mu,\;\;\; V[\bar{X}_n] = \frac{\sigma^2}{n} \]

\subsection{Expected value of the sample variance}
Let $S'^2_n = \frac{1}{n-1}\sum_{i=1}^{n} (X_i-\bar{X})^2 $ be the sample
variance. Then, it holds,
\[ \mathbb{E}[S'^2_n] = \sigma^2 \]

\subsection{Logarithmic Distribution Transformations}
\todo[inline]{Review all this}
If we take a constant $k$ and perform a logarithmic transformation of the sample
distribution $\bar{X}_n$ with this constant. For a fixed $k$ it holds,
\[ \log(\bar{X}_n + k) \sim \mathcal{N}(\mu, \sigma^2) \]

\subsection{Sample Mean of Normal Distribution}
Let $(X_1,\dots,X_n)$ a s.r.s. of size $n$ of a r.v.
$\mathcal{N}(\mu,\sigma^2)$. Then, the \textbf{sample mean} $\bar{X}_n =
\frac{1}{n}\sum_{i=1}^n X_i$ satisfies,
\[ \bar{X} \sim \mathcal{N}(\mu,\frac{\sigma^2}{n}) \]

\subsection{Z statistic or Standardization}
The $Z$ statistic is obtained when the sample mean is standardized,
\[ Z = \frac{\bar{X}_n - \mu}{\sigma / \sqrt{n}} \sim \mathcal{N}(0,1) \]

\subsection{Chi-Square Distribution}
A Chi-square distribution $\mathcal{X}_n^2$ with $n$ degrees of freedom, is the
distribution that gives the sum of $n$ independent r.v. $Z_1,\dots,Z_n$ all of
them with $\mathcal{N}(0,1)$ distributions. Thus,
\[ \sum_{i=1}^n Z_i^2 \sim \mathcal{X}_n^2 \] and it holds that $\mathcal{X}^2_n
= $Gamma$(n/2,2)$

\subsection{Fisher's Theorem}
Let $(X_1,\dots,X_n)$ be a s.r.s of r.v. that follows
$\mathcal{N}(\mu,\sigma^2)$. Then,
\[ \frac{(n-1)S'^2_n}{\sigma^2} = \sum_{i=1}^{n} (\frac{X_i -
\hat{X}_n}{\sigma})^2 \sim \mathcal{X}_{n-1}^2 \] and
\begin{center}
    $\hat{X}_n$ and $\mathcal{X}_n^2$ are independent
\end{center}

\subsubsection{Dregrees of freedom remark}
\todo[inline]{diapo 2.21}

\subsection{Student's Distribution}
Let $X \sim \mathcal{N}(0,1)$ and $Y \sim \mathcal{X}_v^2$ be independent. The
distribution of the r.v. 
\[ T = \frac{X}{\sqrt{Y/v}} \] is a Student's $t_v$ with $v$ degrees of freedom

\subsection{T statistic}
Let $(X_1,\dots,X_n)$ a s.r.s. of a r.v. that follows a
$\mathcal{N}(\mu,\sigma^2)$. Then, a $T$ statistic is,
\[ T = \frac{\bar{X_n}-\mu}{S'^2_n / \sqrt{n}} \sim t_{n-1}\]

\subsection{Snedecor's Distribution}
Let $X_1 \sim \mathcal{X}_{v_1}^2$ and $X_2 \sim \mathcal{X}_{v_2}^2$ be
independent. The distribution of the r.v.
\[ F = \frac{X_1/v_1}{X_2/v_2} \] is a Snedecor's $F$ distribution with $v_1$
degrees of freedom in the numerator, and $v_2$ degrees of freedom in the
denominator. It's denoted $\mathcal{F}_{v_1,v_2}$

\subsubsection{Note:}
\[ F_{1,v} = t^2_v \]

\subsection{F statistic}
Let $(X_1,\dots,X_{n_1})$ be a s.r.s. of a $\mathcal{N}(\mu_1,\sigma^2_1)$ with
sample variance $S'^2_n$. Let $(Y_1,\dots,Y_{n_2})$ be a different s.r.s. of a
$\mathcal{N}(\mu_2,\sigma^2_2)$ with sample variance $S'^2_2$. Let the first and
the second distributions be independent. Then,
\[ F = \frac{S'^2_1/\sigma^2_1}{S'^2_2/\sigma^2_2} \sim
\mathcal{F}_{n_1-1,n_2-1}\]

\section{Large Sample Inference}
What happens when we can't ensure that the underlying distribution is a normal?

\subsection{Convergence in distribution}
The sequence of r.v. converges in distribution to the r.v. $X$ if,
\[ \lim_{x\rightarrow\infty}F_{X_n}(x) = F_X(x) \] for all the points $x$ where
$F_X(x)$ is continuous. This is denoted as,
\[ X_n \xrightarrow{d} X \]  

\subsection{Central Limit Theorem}
\todo[inline]{explain much better this theorem}
Let $X_1,\dots,X_n$ i.i.d. r.v. with expectation $\mathbb{E}[X_i] = \mu$ and
variance $V[X_i]=\sigma^2 < \infty$. Then, the c.d.f. of the r.v. converges to
the c.d.f of a $\mathcal{N}(0,1)$ as long as $n \rightarrow \infty$. This is,
\[ Z_n = \frac{\bar{X}_n - \mu}{\sigma/\sqrt{n}} \xrightarrow{d}
\mathcal{N}(0,1)\]

\begin{tcolorbox}
    \textbf{Summary}\\
    \[
    (X_1,\dots,X_n)
    \begin{cases}
        \sim \mathcal{N}(\mu, \sigma^2) \Rightarrow \bar{X}_n \sim \mathcal{N}(\mu, \sigma^2/n)\\
        \nsim \mathcal{N}(\mu, \sigma^2)
            \begin{cases}
                n < 30 \Rightarrow \text{?}\\
                n \geq 30 \Rightarrow \bar{X}_n \cong \mathcal{N}(\mu, \sigma^2/n)
            \end{cases}
    \end{cases}
    \]
\end{tcolorbox}
In general, when inference, if the original distribution follows a normal and
the sample size is small, we need to use the Student's t distribution.
Otherwise, use the normal. \todo{explain better}


\section{Properties of estimators}
\subsection{Biased and Unbiased}
Given an estimator $\hat{\theta}$ the quantity 
\[ \textbf{B}[\hat{\theta}] = \mathbb{E}[\hat\theta] - \theta \] is the bias of
the estimator. It is said that a estimator with zero bias is unbiased.
$\mathbb{E}[\hat\theta] = \theta$
\missingfigure{two explanatory dartboard}

\subsection{Estimation Error}
Once observed the value of $\hat{\theta}$ of a sample the estimation error is
the quantity $\hat{\theta}_n - \theta$. Note that the estimation error is
usually different from zero. The bias is the mean estimation error.
\[ \textbf{B}[\hat{\theta}] = \mathbb{E}[\hat{\theta}_n] - \theta =
\mathbb{E}[\hat{\theta}_n - \theta] \]

\subsubsection{Error measures}
There are lots of error measurements and each one is used for different
implications. The most commons are:
\begin{itemize}
    \item Mean Absolute Error: $\text{MSE} = \mathbb{E}[|\hat{\theta}_n -
    \theta|]$
    \begin{tcolorbox}
        Note that \[ \text{MSE} = \textbf{B}^2(\hat{\theta}_n) +
        V(\hat{\theta}_n) \]
    \end{tcolorbox}
    \item Mean Square Error: $\text{MAE} = \mathbb{E}[(\hat{\theta}_n -
    \theta)^2]$
\end{itemize}
\todo[inline]{Add more error measurements and his implications}

\subsubsection{Estimator selection}
When we have to choose between two estimators $\hat{\theta}_{n,1}$ or
$\hat{\theta}_{n,2}$
\begin{itemize}
    \item If both are unbiased, choose the one with smallest variance.
    \item If at leas is biased, then choose the one with smallest error (e.g.
    MSE).
\end{itemize}

\subsection{Relative Error}
One problem that appears when measuring errors is that in general one error cant
be directly compared with another one. One way to overcome this is to use a
measurement that holds the same scale.
\subsubsection{Coefficient of Variation}
Given a estimator $\hat{\theta}$ with standard deviation $\sigma(\hat{\theta})$.
The coefficient of variation or relative standard deviation (RSD) is defined as
\[ \text{C.V.}(\hat{\theta}) =
\frac{\sigma(\hat{\theta})}{\hat{\theta}}\text{x}100\;\;\;\;\;\;
\text{if}\hat{\theta} \text{ is unbiased}\]
\[
\text{RRMSE}(\hat{\theta}) = \text{C.V.}(\hat{\theta}) = 
\frac{\sqrt{\text{MSE}(\hat{\theta})}}{\hat{\theta}}\text{x}100\;\;\;\;\;\; \text{if
}\hat{\theta} \text{ is biased}
 \] 

\subsection{Invariant}
\todo[inline]{add more invariant properties}
\subsubsection{Translation Invariant}
It is said that a estimator is translation invariant when it satisfies
\[ \hat{\theta}(X_1 + c,\dots,X_n + c) = \hat{\theta}(X_1,\dots,X_n) +
c,\;\;\;\; \forall c \in \mathbb{R}\]
\todo{example}

\subsubsection{Scale Invariant}
It is said that a estimator is scale invariant when it satisfies
\[ \hat{\theta}(cX_1,\dots,cX_n) = c\hat{\theta}(X_1,\dots,X_n) ,\;\;\;\;
\forall c > 0 \]
\todo{example}

\subsection{Consistency}
The idea of consistency is related with the size of the sample. This property
apply when the probability that the estimator $\hat{\theta}$ decay from the real
$\theta$ when $n\rightarrow\infty$. More formally \todo{add explanatory plot}
\[ P(|\bar{X}_n - \mu|) > 1 \;\;\text{ when }\;\; n\rightarrow\infty\]

\subsubsection{Weak Consistency}
Let $X$ be a r.v. with induced probability $P( \cdot ) = P(\cdot\;;\theta)$. Let
$(X_1,\dots,X_n)$ be a s.r.s of $X$, and let $\hat{\theta}_n =
\hat{\theta}_n(X_1,\dots,X_n)$ be an estimator of $\theta$. The sequence
$\{\hat{\theta}_n,\; n \in \mathbb{N}\}$ is consistent (or weak consistent, or
consistent in probability) for $\theta$, denoted
$\hat{\theta}\xrightarrow{P}\theta$, if and only if,
\[ \lim_{n\rightarrow\infty} P(|\hat{\theta}_n - \theta| > \epsilon)=0,\;\;\;
\forall \epsilon>0\]

\subsubsection{Consistency in squared mean}
The following definition is stronger than the previous one,\\
A sequence of estimators $\{\hat{\theta}_n, n \in \mathbb{N}\}$ is consistent in
squared mean for $\theta$, denoted $\hat{\theta}\xrightarrow{sq.m.}\theta$, if
it verifies
\[ \lim_{n\rightarrow\infty} \text{MSE}(\hat{\theta}_n) = 0 \] or equivalently,
\[ \lim_{n\rightarrow\infty} \textbf{B}(\hat{\theta}_n) = 0 \;\;\;
\text{and}\;\;\; \lim_{n\rightarrow\infty} V(\hat{\theta}_n) = 0 \] This theorem
also states that,
\[ \hat{\theta}_n \xrightarrow{sq.m.}\theta \Rightarrow \hat{\theta}_n
\xrightarrow{P}\theta \]

\section{Law of Large Numbers}
The law of large numbers (LLN) is along with the central limit theorem the two
key laws in probability. It states that when the estimator is consistent and the
sample size is enough big, the sample mean equals the population mean. More
formally,
\subsubsection{Definition of LLN}
Let $(X_1,\dots,X_n)$ be a s.r.s. of a r.. $X$ with mean $\mu$ and variance
$\sigma^2<\infty$. Then,
\[ \bar{X}_n \xrightarrow{P} \mu \]

\section{Algebra of Consistency}
It is stated that any continuous transformation of a consistent estimator is
consistent for the same transformation of the parameter. We can define formally
in two ways,\\

\subsubsection{Single estimator definition}
Let $\hat{\theta}\xrightarrow{P}\theta$ and let $g(x)$ be a continuous at
$x=\theta$. Then $g(\hat{\theta}_n) \xrightarrow{P}g(\theta)$
\subsubsection{Multiple estimator definition}
Let $\hat{\theta}\xrightarrow{P}\theta$ and
$\hat{\theta}'\xrightarrow{P}\theta'$. Let $g(x,y)$ be a continuous at
$(x,y)=(\theta,\theta')$. Then $g(\hat{\theta}_n,\hat{\theta}_n')
\xrightarrow{P}g(\theta,\theta')$

\subsubsection{Corollary}
From this two definitions we can summarize multiples algebras between
estimators.
\begin{itemize}
    \item $\hat{\theta}_n + \hat{\theta}_n' \xrightarrow{P} \theta + \theta'$
    \item $\hat{\theta}_n\hat{\theta}_n' \xrightarrow{P} \theta\theta'$
    \item $\hat{\theta} / \hat{\theta}_n' \xrightarrow{P} \theta/ \theta' \;\;\;
    \text{if } \theta' \neq 0$
    \item $\sqrt{\hat{\theta}_n} \xrightarrow{P} \sqrt{\theta} \;\;\;\;
    \text{if}\; P(\hat{\theta}_n \geq 0) = 1$
    \item $a_n\hat{\theta}_n \xrightarrow{P} a\theta \;\;\; \text{with} a_n$
    being a sequence of constants. 
\end{itemize}

\subsection{Slutsky's Theorem}
Let $U_n$ and $W_n$ be random sequences satisfying
\[ U_n\xrightarrow{d}\mathcal{N}(0,1),\;\;\;\;\; W_n\xrightarrow{P}1\] Then, it
holds,
\[\frac{U_n}{W_n} \xrightarrow{d} \mathcal{N}(0,1)\]
\todo[inline]{Add asymptotic normality of T stat example}

\section{Fisher's Information}
Let $X$ be a continuous r.v. with distribution that depends of $\theta \in
\Theta \subset \mathbb{R}$. The Fisher's information of $X$ over $\theta$ is
defined as
\[ I(\theta) = \mathbb{E}\left[\left(\frac{\partial\log
f(X;\theta)}{\partial\theta}\right)^2\right] \] for continuous distributions.
For discrete distributions the definition is the same but replacing the p.d.f
for the p.m.f (replacing $f(X;\theta) \text{ by } p(X;\theta)$)
\begin{tcolorbox}
    Note that
    \[ \frac{\partial\log f(X;\theta)}{\partial\theta} = \frac{\frac{\partial
    f(X;\theta)}{\partial\theta}}{f(X;\theta)} \]
\end{tcolorbox}

This quantity is the relative variation rate of $f$ when varying $\theta$, for a
realized value $x$ of $X$. It represents the information of $x$ to discriminate
$\theta$ from a near value $\theta+h$. Fisher’s information is the mean
information of the r.v. $X$ about $\theta$.
\todo[inline]{add better and conceptual explanation. Maybe an image too}

\subsection{Fisher's Information of a sample}
The Fisher's information of a s.r.s $(X_q,\dots,X_n)$ of a continuous r.v. $X$
over $\theta$ is defined as
\[ \mathcal{I}_n(\theta) = \mathbb{E}\left[\left(\frac{\partial\log
f(X_1,\dots,X_n;\theta)}{\partial\theta}\right)^2\right] \] For $X$ discrete
just replace $f(\cdot;\theta)$ by the joint p.m.f. $p(\cdot;\theta)$

\subsubsection{Fisher's equality}
\todo[inline]{search and add information about this equality}
Under certain regularity assumptions,
\[ \mathcal{I}_n(\theta) = nI(\theta) \]

\subsubsection{Frechet-Cr\'amer-Rao lower bound}
\todo{Check and write this assumptions}Under certain regularity assumptions, any
unbiased estimator $\hat{\theta}_n=\hat{\theta}_n(X_1,\dots,X_n)$ of $\theta$
satisfies
\[ V(\hat{\theta}_n) \geq 1 / \mathcal{I}_n(\theta)\]

\subsection{Efficient estimator}
An unbiased estimator $\hat{\theta}_n$ of $\theta$ that satisfies
$V(\hat{\theta}_n)=1 / \mathcal{I}_n(\theta)$ is called efficient. If an
estimator is efficient it means that it is the best possible estimator.
\todo{add example}

\section{Robust Estimators}
\todo[inline]{search more info about robust estimators and complete this section}
In real problems it is common that the data contains some contamination in form
of measurements errors or other problems. In statistics it is said that an
estimator $\hat{\theta}$ is robust if it preserves good properties (small bias
and variance) even if the sample is contaminated.

The theory of statistical robustness is deep. For this, we are going to use this
widely-used contamination model for $f(\;\cdot;\theta)$.
\[ (1-\epsilon)f(x;\theta) + \epsilon g(x),\;\;\;\; x \in \mathbb{R},\\
\text{ with } 0<\epsilon<0.5 \\ \text{ and arbitrary p.d.f }\;\; g\]

\subsection{Outliers}
\todo[inline]{rewrite this whole part, because tis copy-paste from another book}
The concept of outlier is intimately related with robustness. Outliers are
“abnormal” observations in the sample that seem very unlikely for the assumed
distribution model or are remarkably different from the rest of sample
observations. Outliers can be originated by measurement errors, exceptional
circumstances, changes in the data generating process, etc.

There are two main approaches for preventing outliers or contamination to
undermine the estimation of $\theta$:
\begin{enumerate}
    \item Detect the outliers through a diagnosis of the model fit and
    re-estimate the model once the outliers have been removed.
    \item Employ a robust estimator
\end{enumerate}

The first approach is the traditional one and is still popular due to its
simplicity. Besides, it allows us to employ non-robust efficient estimators that
tend to be simpler to compute, provided the data has been cleared adequately.
However, this procedure may quickly run into problems, since, for example,
detecting outliers in higher dimensions is usually complicated and this
detection may require manual inspection of the data.

In addition, robust estimators may be needed even when performing the first
approach, as the following example illustrates. A simple rule to detect outliers
in a normal population is to flag as outliers the observations that lie further
away than $3\sigma$ from the mean $\mu$, since those observations are highly
extreme. Since their probability is $0.0027$, we expect to flag as an outlier
$1$ out of $371$ observations if the data comes from a perfectly valid normal
population. However, applying this procedure entails estimating first $\mu$ and
$\sigma$ from the data. But the conventional estimators, sample mean and
variance, are also very sensitive to outliers, and therefore their resulting
values may hide the existence of outliers. Therefore, it is better to rely on a
robust estimator, which brings us back to the second approach. As a consequence,
it is sometimes preferred to employ robust estimators from the beginning.

The next definition introduces a simple measure of the robustness of an
estimator.

\subsection{Finite-sample breakdown point}
The breakdown point of an estimator can be interpreted as the maximum fraction
of the sample that can be changed without modifying the value of $\hat{\theta}$
to an arbitrary large value.

\todo[inline]{add better explanation, examples of robust estimators, trimmed mean, etc}

\section{Estimation methods}
\todo[inline]{add introductory text}
\subsection{Method of moments}
\subsubsection{Population moments}
Let's consider a population $X$ with a unknown distribution that depends on $K$
unknown parameters $\theta_1,\dots,\theta_K$. The populations moments are
functions of the unknown parameters, if they exists. Formally,
\[ \alpha_r = \alpha_r(\theta_1,\dots,\theta_K) = \mathbb{E}[X^r],\;\;\;
r=1,2,3,\dots \]

\subsubsection{Sample moment}
Given a s.r.s. of $X$, it's denoted by $a_r$ to the sample moment of order $r$
that estimate $\alpha_r$
\[ a_r = \bar{X}^r = \frac{1}{n}\sum_{i=1}^n X^r_i\;\;\;\;\;r=1,2,3,\dots\] It's
important to note that the sample moments do not depend of
$\alpha_1,\dots,\alpha_K$ but the population moment does. 

\subsubsection{Method of Moments}
Let $X$ be a r.v. that follow a unknown distribution of unknown parameters
$\theta_1,\dots,\theta_K$. The method of moments estimates this parameters by
resolving a system of equations of shape
\[ \alpha_r(\theta_1,\dots,\theta_K) = a_r\;\;\;\; r=1,\dots,R \] where $R \geq
K$ is the lowest integer such that the system admits a unique solution. This
estimator of $\theta$ is denoted by $\hat{\theta}_{MM}$
\todo[inline]{add example}

\subsubsection{Consistency of moments estimators}
In the moment estimator method, if $\mathbb{E}[(X^{r_k}-\alpha_{r_k})^2]<\infty$
and $\theta_k=g_k(\alpha_{r_1},\dots,\alpha_{r_K})$, with $g_k$ continuous and
$k=1,\dots,K$ then
\[ \hat{\theta}_k = g_k(a_{r_1},\dots,a_{r_K})\xrightarrow{P}\theta_k, \;\;\;\;
i=1,\dots,K \]

\subsection{Maximum Likelihood Method}
\todo[inline]{introductory example with unfair toss coin 0.8,0.2}
This method relies in the believe that whatever happens in the reality is what
is more probably to happen.

The Maximum Likelihood estimator (MLE) denoted by $\hat{\theta_{\text{MLE}}}$ of
$\theta$ given the realized sample $(X_1=x_1,\dots,X_n=x_n)$ is the value
$\hat{\theta}$ that maximizes the likelihood
\[ L(\hat{\theta};x_1,\dots,x_n) = \max_{\theta\in\Theta}L(\theta;x_1,\dots,x_n)
\]
\[ \hat{\theta}_{\text{MLE}} = \arg \max_{\theta\in\Theta}
L(\theta;x_1,\dots,x_n) \]
\begin{tcolorbox}
    Note that the maximum $\hat{\theta}$ of the likelihood is also the maximum
    of the loglikelihood
    \[ \ell(\hat{\theta};x_1,\dots,x_n) = \log L(\theta;x_1,\dots,x_n)\] This is
    because the logarithm is monotonously increasing.
\end{tcolorbox}
It's important to remark that when calculating the MLE sometimes is more easy to
just compute the logMLE.

\subsubsection{Computation of MLE}
In order to find that maximum likelihood, there are some remarks to take into
account. If $\Theta$ is finite, the maximum can be found evaluating all possible
values of $L(\theta;x_1,\dots,x_n)$. If $\Theta$ is infinite and
$L(\theta;x_1,\dots,x_n)$ is differentiable respect $\theta$, then the solutions
of the likelihood equations can be calculated
\[
\frac{\partial}{\partial\theta_k}L(\theta;x_1,\dots,x_n)=0,\;\;\;\;k=1,\dots,K
\] Later, and because the equations can have multiples solutions (local minimas,
and global minimas) we need to compare the solutions among themselves and with
the boundaries values of $\Theta$ to find the global maximum.

\todo[inline]{example MLE under Normal}

\subsubsection{Properties}
\begin{itemize}
    \item The MLE of $\theta$ is not necessarily unbiased.
    \item The MLE of $\theta$ is not necessarily unique.
    \item If an unbiased and efficient estimator of $\theta$ exists, then that
    estimator is the unique MLE of $\theta$
    \item The MLE is invariant with respect to one-to-one transformations of the
    parameter, that is, if $\omega=h(\theta)$ where $h$ is one-to-one and
    $\hat{\theta}$ is the MLE of $\theta$, then $\hat{\omega}=h(\hat{\theta})$
    is the MLE of $\omega$
\end{itemize}

\subsubsection{Asymptotic efficency of MLEs}
Let $f(x;\theta)$ be a p.d.f. (or p.m.f.) of a r.v. X, where $\theta\in\Theta$.
Let $\Theta$ be a open interval from $\mathbb{R}$. Under certain regularity
conditions\todo{check this conditions}, it holds that any sequence
$\hat{\theta}_n$ of solutions of the likelihood equations that is consistent for
$\theta$ verifies
\[ \sqrt{n}(\hat{\theta}_n-\theta) \xrightarrow{d} \mathcal{N}(0,
I(\theta)^{-1}) \] This definition can be used to calculate the MLE when a
parameter $\theta$ is unknown.
\todo[inline]{example}