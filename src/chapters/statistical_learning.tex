\section{Introduction - Supervised Learning Framework}
\todo[inline]{rewrite all this using all the content of the slides but without being a scheme}
The usual framework used in machine learning is DGP (data generating process?)\todo{research about this}
\begin{center}
    Data = Model + Noise
\end{center}

or in a statistical point of view, we assume $g(x)$ as the true model of the data
\begin{equation*}
    y = g(x_1,\dots, x_k) + \text{Noise}
\end{equation*}
If we use this equation for a linear approximation
\begin{align*}
    \text{Statistical approach: }& y = \beta_0 + \beta_1 x_1 + \dots + \beta_p x_p + \epsilon\\
    \text{Machine Learning approach: }& y = \text{map}(x_1,\dots,x_p) + \epsilon
\end{align*}

The two main categories of problems are named as, \textbf{classification and regression.}

\subsection{Dimensionality Curse}
\todo[inline]{write dimensionality curse problem}

\subsection{Prediction Error}
In predictive models there are three sources of uncertainty
\begin{enumerate}
    \item Estimation error: The error in the coefficients when the estimation is true.
    \item Model Bias: The error in the linear approximation when the true model
    is non-linear, or contains other variables.
    \item Irreducible Error: The noise in the data.
\end{enumerate}
\begin{equation*}
    (\text{Prediction Error}^2) = \sigma^2 + \text{Bias}^2 + \text{Var}
\end{equation*}

\subsubsection{Statistics}
It focus on minimizing bias (by assuming knowledge of the population). By doing
so, it is able to get formulas for the Var that provides explanation. The Var can be large in practice.

\subsubsection{Machine Learning}
It focus on minimizing $\text{Bias}^2 + \text{Var}$. No assumptions needed,
hence no formulas and no direct explanation. Best predictive performance.

\missingfigure{bias vs variance image}

\subsubsection{Overfitting}
\todo[inline]{write this}
\missingfigure{example plot of a training showing the diverge between train and test loss}

\subsection{Classification Problems}
\todo[inline]{write extensively about this}

Some well-know statistical classification models are:
\begin{enumerate}
    \item Logistic regression
    \item Bayes classifiers
    \begin{enumerate}
        \item LDA
        \item QDA
        \item Naive Bayes
        \item Shrinkage classification
    \end{enumerate}
\end{enumerate}

Some well-know machine learning classification models are:
\begin{enumerate}
    \item Nearest Neighbors
    \item Neural Networks
    \item Support Vector Machines
    \item Decision Trees, Random Forest, Gradient Boosting
\end{enumerate}

\section{Probabilistic Learning}
This approach aims to first predict the probabilities of a observation to belong
to each of the categories. Later, select the most probable as the correct category.

They are two main families:
\begin{enumerate}
    \item Bayes Classifiers
    \item Logistic Regression
\end{enumerate}

\subsection{Bayes Classifiers}
Given the predictors $y | x_1,\dots,x_p$ focus on the conditional probability
$p(y | x_1,\dots,x_p)$.

In the bayes approach,  this is modeled in an indirect way:
\begin{enumerate}
    \item First model the predictors $X$ separately for each given class $y$: $p(x|y)$
    \item Second, apply the bayes formula to get $p(y|x)$ 
\end{enumerate}

This approach is more stable than logistic regression when the classes are well
separated. It performs well when the variables are gaussian distributed.

\subsubsection{Classifier}
Let $g=\{1,2,\dots,G\}$ a set of labels, let $\pi_g =P(y \in g)$ denote the
prior probability of $y$ to belong to $g$. Let $g : f_g(x)$ the
multivariate distribution of predictors to model. Thus,
\begin{equation}
    p_g(x) = P(y \in g | X = x) = \frac{f_g(x)\pi_g}{\sum_k f_k(x)\pi_k}
\end{equation}
Finally, the maximum probability will be the ones to assign, $\max_g\;\; f_g(x)\pi_g$. 
\begin{tcolorbox}
    Note: The posterior probabilities are the same as the logistic regression ones. 
\end{tcolorbox}
The bayes classifiers are optimal in the sense they minimize the classification
error rate. In practice $\pi_g$ and $f_g(x)$ needs to be estimated, in order to do that using
the proportion of training observations to belongs to class $g: \hat{\pi}_g
\frac{n_g}{n}$. Whiout previous knoledge, $\hat{\pi_g} = 1/G$ is usually used.

To estimate $f_g(x)$ it is usually assumed to follow a multivariate normal, $f_g
\sim \mathcal{N}(\mu_g, \Sigma_g)$. Using bayes rule we have,
\begin{equation}
    \max_g \;\; f_g(x)\pi_g = \min_g \;\; (x-\mu_g)^T \Sigma_g^{-1} (x-\mu_g) + \log \det(\Sigma_g)-2\log\pi_g  
\end{equation}
\todo[inline]{investigate about mahalanobis distance}

This is called \textbf{quadratic discriminant analysis (QDA)}. 

\subsection{Quadratic discriminant analysis (QDA)}
