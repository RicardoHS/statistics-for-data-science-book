\section{Introduction}
Regression modeling is a set of statistical tools that aims to model associations
rules between variables. This models can be later use to predict new observations,
system explanation, variable screening or parameter estimation.

It's important to note the difference between correlation and regression. In correlation
the relationship is not directional, so, it's interest is only on how they are mutually
associated. In regression the interest come from how one variable respond to others.

\section{Linear Regression}
\subsection{Linear Regression Models}
Problems where there is only one independent variable (regressor or covariate) is called 
\textbf{simple linear regression model}. For these problems, the approach is to try to 
estimate the mean values of a variable respect to other $\mathbb{E}[Y|x]$, more formally
\begin{equation}
    Y = \beta_0 + \beta_1 x + \epsilon
\end{equation}
where $\epsilon$ is an error term.
\missingfigure{simple linear regression problem, ex: birth rate/poverty index}

A regression model that contains more than one regressor variable is called
\textbf{multiple linear regression model}. The model is similar to the simple
one, they have in common a \textbf{response variable} $Y$, an \textbf{intercept}
$beta_0$ and an \textbf{error term} $\epsilon$. Contrary to the simple one, the multiple linear
regression model has \textbf{more than one covariate or regressor} $x_{ij}$.
\begin{gather*}
    Y_1 = \beta_0 + \beta_1 x_{11} + \beta_2 x_{12} + \dots + \beta_k x_{1k} + \epsilon_1\\
    Y_2 = \beta_0 + \beta_1 x_{21} + \beta_2 x_{22} + \dots + \beta_k x_{2k} + \epsilon_2\\
    \vdots\\
    Y_n = \beta_0 + \beta_1 x_{n1} + \beta_2 x_{n2} + \dots + \beta_k x_{nk} + \epsilon_n\\
\end{gather*}

\subsubsection{Matrix Formulation}
The above model can be expressed in matrix equation (note the bold font):
\begin{equation}
    \bm{Y} = \bm{X}\bm{\beta} + \bm{\epsilon}
\end{equation}
where
\begin{equation}
    \bm{Y} = 
    \begin{bmatrix}
        y_1\\y_2\\\vdots\\y_n\\
    \end{bmatrix}
    \bm{X} = 
    \begin{bmatrix}
        1 & x_{11} & \dots & x_{1k}\\
        1 & x_{21} & \dots & x_{2k}\\
        & & \vdots & \\
        1 & x_{n1} & \dots & x_{nk}\\
    \end{bmatrix}
    \bm{\beta} =
    \begin{bmatrix}
        \beta_0\\\beta_2\\\vdots\\\beta_n\\
    \end{bmatrix}
    \bm{\epsilon} = 
    \begin{bmatrix}
        \epsilon_1\\ \epsilon_2 \\ \vdots \\ \epsilon_n\\
    \end{bmatrix}
\end{equation}

Each column of $\bm{X}$ contains a particular covariate, is assumed to be know.
$\bm{\beta}$ is the vector of unknown parameters to be estimated from the model and
$\bm{Y}$ and $\bm{\epsilon}$ are random vectors whose elements are random variables.

\subsection{Assumptions}
These are the assumptions to be aware of when using linear regression models.
\begin{itemize}
    \item Linearity: The mean of the response is a linear function of the predictors:
    \[ \mathbb{E}[\bm{Y}|\bm{X}_1=x_1,\dots,\bm{X}_k=x_k] = \beta_0 + \beta_1 x_1 + \dots + \beta_k x_k \]
    \item Independence: The errors $\epsilon_i$ are independent,
    \[ \text{Cov}(\epsilon_i, \epsilon_j) = 0, \;\;\; i \neq j \] 
    \item Homocedasticity: The variance of the errors $\epsilon_i$ at each value of $x_i$ is constant.
    \[ \text{Var}[\epsilon_i | \bm{X}_1] = x_1,\dots, \bm{X}_k=x_k] = \sigma^2 \]
    \item Normality: The errors of $\epsilon$ are normal distributed
    \[ \epsilon_i \sim \mathcal{N}(0, \sigma^2) \]
\end{itemize}

\begin{tcolorbox}
    In summary
    \begin{equation*}
        \bm{\epsilon} \sim \mathcal{N}(0,\bm{I}\sigma^2) \Rightarrow \bm{Y}\sim \mathcal{N}(\bm{X\beta},\bm{I}\sigma^2) 
    \end{equation*}
\end{tcolorbox}

\subsection{Least Squares}
This approach tries to minimize the residual sum of squares (RSS)
\begin{equation*}
    \text{RSS}(\beta_0,\beta_1) = \sum_{i=1}^n (Y_i - (\beta_0 + \beta_1 X_i ))^2
\end{equation*}

This approach, penalizes more the points that are further from the regression
line. Also, it's computational less expensive than other approaches.

\begin{tcolorbox}
    For the simple linear regression model, the solution for the least squares is 
    \[ \hat{\beta}_1 = \frac{S_{xy}}{S_x^2}, \;\;\;\; \hat{\beta}_0 = \bar{Y}-\hat{\beta}_1 \bar{X} \] 
    where $\bar{X}$ and $\bar{Y}$ are the
    sample mean of $X$ and $Y$. $S_x^2$ is the sample variance of $X$. $S_{xy}$ is the sample covariate between $X$ and $Y$.
\end{tcolorbox}

\subsubsection{Matrix Form}
The least squares for multiple linear regression can be denoted 
\begin{equation}
    \text{RSS}(\beta_0,\beta_1,\dots,\beta_k) = \sum_{i=1}^n (Y_i - (\beta_0 + \beta_1 X_{ik} + \dots + \beta_k x_{ik}))^2
\end{equation}
or in matrix form
\begin{equation}
    \text{RSS}(\bm{\beta}) = (\bm{Y}-\bm{X\beta})'(\bm{Y}-\bm{X\beta})
\end{equation}

\begin{tcolorbox}
Solution: In order to calculate the vector of $\beta_i$ we calculate the
derivate of the matrix
\begin{equation*}
    \frac{\partial \text{RSS}}{\partial\bm{\beta}} = -2\bm{X'Y} + 2\bm{X'X\beta}
\end{equation*}
setting the equation to zero, we obtain
\begin{equation*}
    \bm{\hat{\beta}} = (\bm{X'X}^{-1} \bm{X'Y})
\end{equation*}
Once the parameters are estimated we obtain
\begin{equation}
    \bm{\hat{Y}} = \dots = \bm{HY}
\end{equation}
\begin{equation}
    \bm{\hat{\epsilon}} = \bm{(I-H)Y}
\end{equation}
where $\bm{H}$ is called the \textit{Hat Matrix} and what it does is to project $\bm{Y}$ into the regression hyperplane.
\end{tcolorbox}

\subsubsection{Properties}
\begin{itemize}
    \item The sum of residuals is zero.
    \[ \sum_{i=1}^n \hat{\epsilon}_i = \bm{1'(I-H)Y} = 0 \]
    \item The sum of observed data is equal to the sum of fitted values,
    \[ \sum_{i=0}^n Y_i = \sum_{i=0}^n \hat{Y}_i = \bm{1'\hat{Y}} \]
    \item The residuals are orthogonal to the predictors
    \[ \sum_{i=0}^n x_i\hat{\epsilon}_i = \bm{X'\hat{\epsilon}} = 0 \]
    \item The residuals are orthogonal to the fitted values
    \[ \sum_{i=0}^n \hat{y}_i\hat{\epsilon}_i = \bm{\hat{Y}'\hat{\epsilon}} = 0 \]
\end{itemize}

\subsection{Maximum Likelihood}
